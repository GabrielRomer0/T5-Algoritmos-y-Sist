\documentclass[12pt]{article}
\usepackage[utf8]{inputenc}
\usepackage[spanish]{babel}
\usepackage{amsmath}
\usepackage{geometry}
\usepackage{graphicx}
\usepackage{titlesec}
\usepackage{setspace}
\usepackage{parskip}
\geometry{a4paper, margin=4cm}

\titleformat{\section}{\large\bfseries}{\thesection.}{0.5em}{}

\title{\textbf{\huge La calculadora}}

\author{ Gabriel Romero Raddatz}

\date{Mayo 2025}

\begin{document}

\maketitle

\section{¿Qué es una calculadora y quién la inventó?}

\begin{lef}
    
Una calculadora es un dispositivo hecho para realizar cálculos aritméticos de manera rápida, en la actualidad hasta cálculos complejos con funciones trigonométricas complejas. La Calculadora utiliza algoritmos que han ido desarrollándose con el paso de los siglos. \\
\\
Algunos ejemplos de estos son el cuadrado de bonimio $(a+b)^{2}$  descubierto por Al-Karaji en el siglo XI, probado para cualquier exponente en $\Re$ por Isaac Newton en el siglo XVII y para cualquier término en $\Re$ por John Colson en 1736. 


El concepto de \textbf{"Calculadora} resulta más notorio durante el siglo \textbf{XVII }, donde el Alemán Wilhelm Shickard creó un dispositivo llamado "reloj calculador" en 1623, capaz de sumar y restar números de hasta 6 cifras. \\
La Pascalina, que se dió más a conocer, fue creada por Blaise Pascal en 1642. Se diseñó para facilitar las tareas de cálculo de su padre, un contador. El artefacto podía realizar operaciones de suma, resta y permitía realizar multiplicaciones y divisiones mediante sumas y restas repetidas




\section{Conclusión}
La calculadora es un invento que ha evolucionado desde mecanismos simples hasta sistemas digitales muy complejos. \\
Los algoritmos que las hacen funcionar son el cúmulo de trabajo de muchos científicos y matemáticos, y hoy son herramientas fundamentales en la profesional y cotidiana.

\end{lef}
\end{document}